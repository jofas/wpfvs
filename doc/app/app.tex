\section{Application}

% Introduction {{{
Our goal was to make use of a lot of computing power in
order to train our agent to master an OpenAI Gym
Environment (cmp. \ref{s_openai_gym}). In this chapter we
will document how we tried to achieve this goal with our
distributed application.

% }}}

% The agent {{{
\subsection{The agent}

We programmed our agent as a neural network with the Keras
library, which has an API for high level, high abstraction
neural networks.

Keras uses Tensorflow as its backend for computations and
basically only provides a nicer abstraction of Tensorflow.

% example {{{
\begin{mdframed}[style=codebox]
\begin{lstlisting}[language=Python]
# a small example program using Keras
#
# API documentation at: https://keras.io/

# Sequential is the keras object representing a neural net-
# work
from keras.models import Sequential
# a neural net is comprised of layers connected with each
# other. The Dense object represents a layer
from keras.layers import Dense

# the neural net. This specific neural net has four layers
# (the input (size of 12), two hidden (both 64 artificial
# neurons) and the output layer (size of 4)). The input
# layer does not have to be specified since the input_dim
# parameter of the first layer automatically generates the
# input layer.
model = Sequential([
  Dense(64, activation='relu', input_dim=12 ),
  Dense(64, activation='relu'               ),
  Dense(4,  activation='softmax'            ),
])

# define how the model should learn and some other meta
# information for the training process (learning algorithm,
# optimizer, etc)
model.compile(
        optimizer = 'adam',
        loss      = 'categorical_crossentropy',
        metrics   = ['accuracy']
)

# generate a dummy data set with corresponding labels with
# 1000 entries for training
import numpy as np

data   = np.random.random((1000,12))
# generating the labels (either a 0 or a 1)
labels = np.random.randint(2, size=(1000, 1))

# training the model, iterating 10 times
model.train(data, labels, epochs=10)

# using the neural net to predict the label of a random
# data point
test   = np.random.random((1,12))

model.predict(test)
\end{lstlisting}
\end{mdframed}
% }}}
% }}}

\newpage

\subsection{Architecture}

This chapter will describe in detail how our application
is build, how it works and what we use to achieve
concurrency.

On a higher level our application is devided into two
distinct parts, the Executioner ($E$) and the Worker ($W$).

% Network {{{
\subsubsection{Network}

Instances of both, $E_i$ and $W_i$ communicate over a
message broker, in this case RabbitMQ (cmp.
\ref{s_message_broker}, \ref{s_rabbitmq}).

\begin{figure}[H]
\begin{center}
\begin{tikzpicture}

  \node[inner sep=15pt]
    at (0,0) (rmq) {};

  \node [label=$W_2$, inner ysep=50pt, above=4cm of rmq]
    (w_two) {};
  \draw[-{Stealth[color=white,length=10mm]}]
    (w_two.center) -- (rmq);
  \PCIcon{w_two}{scale=.25}

  \node [label=$W_1$, inner ysep=50pt, left =2cm of w_two]
    (w_one)   {};
  \draw[-{Stealth[color=white,length=10mm]}]
    (w_one.center) -- (rmq);
  \PCIcon{w_one}{scale=.25}

  \node [label=$W_0$, inner ysep=50pt, left =2cm of w_one]
    (w_zero)   {};
  \draw[-{Stealth[color=white,length=10mm]}]
    (w_zero.center) -- (rmq);
  \PCIcon{w_zero}{scale=.25}

  \node [right=2cm of w_two] (w_dots) {\dots};

  \node [label=$W_n$, inner ysep=50pt, right=2cm of w_dots]
    (w_n) {};
  \draw[-{Stealth[color=white,length=10mm]}]
    (w_n.center) -- (rmq);
  \PCIcon{w_n}{scale=.25}


  \node [label={below:$E_2$}, inner ysep=40pt,
    below=4cm of rmq]
    (e_two) {}
    edge[-{Stealth[color=white,length=10mm]}] (rmq);
  \PCIcon{e_two}{scale=.25}

  \node [label={below:$E_1$}, inner ysep=40pt,
    left =2cm of e_two]
    (e_one)   {}
    edge[-{Stealth[color=white,length=10mm]}] (rmq);
  \PCIcon{e_one}{scale=.25}

  \node [label={below:$E_0$}, inner ysep=40pt,
    left =2cm of e_one]
    (e_zero)   {}
    edge[-{Stealth[color=white,length=10mm]}] (rmq);
  \PCIcon{e_zero}{scale=.25}

  \node [right=2cm of e_two] (e_dots) {\dots};

  \node [label={below:$E_m$}, inner ysep=40pt,
    right=2cm of e_dots]
    (e_n) {}
    edge[-{Stealth[color=white,length=12mm]}] (rmq);
  \PCIcon{e_n}{scale=.25}


  \node[label={below:RabbitMQ},inner ysep=15pt]
    at (0,0) {};
  \RabbitMQ{rmq}{scale=.25}

\end{tikzpicture}
\end{center}
\caption{Architecture of our Application on the network
  level}
\end{figure}

% }}}

% Executioner {{{
\subsubsection{Executioner}

% }}}

% Worker {{{
\subsubsection{Worker}

here comes the data sanitation part

% }}}

\subsubsection{Queues}


\subsection{Results}

\subsection{Where to go}

(dynamic data generation, loadbalancing, further optimizations
like thread pool exec in worker (not spawning so often),
destroying the bottleneck in executioner (data parsing))
