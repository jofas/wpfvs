\subsubsection{Executor}
\label{s_executor}
\index{Application!Executor}

The Executor $E$ is the part of the application that is
doing the training and testing of our agent. Training and
testing is done incrementally in a loop, the "TTSL"
(Training-Testing-Sending-Loop). An Executor instance
$E_i$ gets provided with its training data from the Worker
instances $W_j$ it is connected to (via the RabbitMQ).

If the testing part of the "TTSL" failed $E_i$ sends the
newly trained agent to every $W_j$ (the sending part of the
"TTSL"). Else if the testing was successful and the agent
mastered the environment (cmp \ref{s_openai_gym}) it sends
a message that testing was successful which kills every
$W_j$.

The Executor is composed of three processes, the main
process, the meta process and the TTSL process.

\begin{itemize}[label={}]

  \item \textbf{main process:}

        The main process first initializes global shared
        variables and constants it shares with the other
        two processes. Then it starts the meta process and
        the TTSL process.

        After that the main process becomes a listener
        which listens for incoming data from the Worker
        instances connected to the RabbitMQ. If new data
        comes in it is saved in a shared variable so the
        TTSL process can access it.
        \index{Application!Executor!main process}

  \item \textbf{TTSL process:}

        The process executing the TTSL.
        \index{Application!Executor!TTSL process}

  \item \textbf{meta process:}

        This process is a listener (like main becomes after
        initializing the shared memory and starting this
        and the TTSL process) which listens to a queue
        of the RabbitMQ so this Executor can communicate
        with the Workers it is connected to.
        \index{Application!Executor!meta process}

\end{itemize}

\begin{figure}[H]
\begin{center}
\begin{tikzpicture}

  \UMLActivitySwimlane{TTSL}{\linewidth/3}{\textheight-5cm}
    {}
  \UMLActivitySwimlane{main}{\linewidth/3}{\textheight-5cm}
    {right=0 of TTSL}
  \UMLActivitySwimlane{meta}{\linewidth/3}{\textheight-5cm}
    {right=0 of main}



  \UMLActivityInitialNodeRelativeTo{above=.5 of main}


  % main {{{
	\UMLActivityStateRelativeToAlterName
    {above=-3 of main}
    {Initialize global shared variables and constants}
    {init}
    {text width=4cm}

  \UMLActivityConcurrentNodeHRelativeTo
    {below=1 of init}
    {c}
    {16cm}

	\UMLActivityStateRelativeToAlterName
    {below=1 of c}
    {Listening for new training data}
    {data}
    {text width=4cm}

  \UMLActivityCentralBufferRelativeToAlterName
    {below=1 of data}
    {Training data set}
    {data_set}
    {text width=4cm}
  % }}}

  % ttsl{{{
  \UMLActivityDescisionNodeRelativeTo
    {above=-6 of TTSL}
    {d_tr}

  \UMLActivityStateRelativeToAlterName
    {below=1 of d_tr}
    {Training}
    {tr}
    {text width=4cm}

  \UMLActivityStateRelativeToAlterName
    {below=1 of tr}
    {Testing}
    {te}
    {text width=4cm}

  \UMLActivityDescisionNodeRelativeTo
    {below=1 of te}
    {d_ttsl}

  \UMLActivityStateRelativeToAlterName
    {below=1 of d_ttsl}
    {Send agent to Worker}
    {saw}
    {text width=4cm}

  \UMLActivityStateRelativeToAlterName
    {below=1 of saw}
    {Send done message to Worker}
    {sdmw}
    {text width=4cm}

  \UMLActivityStateRelativeToAlterName
    {below=1 of sdmw}
    {Write protocol to file}
    {wptf}
    {text width=4cm}

  \UMLActivityExitNodeRelativeTo
    {below=1 of wptf}
    {kill}
  % }}}

  % meta {{{
  \UMLActivityDescisionNodeRelativeTo
    {above=-6 of meta}
    {d_meta}

  \UMLActivityStateRelativeToAlterName
    {below=1 of d_meta}
    {Listening to META\-QUEUE}
    {lmq}
    {text width=4cm}

  \UMLActivityDescisionNodeRelativeTo
    {below=1 of lmq}
    {d_meta_one}

  \UMLActivityStateRelativeToAlterName
    {below=1 of d_meta_one}
    {Send environment to Worker}
    {sew}
    {text width=4cm}

  \UMLActivityStateRelativeToAlterName
    {below=1 of sew}
    {Add message to the protocol}
    {mtp}
    {text width=4cm}
  % }}}

  % conns {{{
	\UMLActivityControlFlow
    {initial.east}{init.north}
    {-|(8,10)|-($(init.north)+(0,0.3)$)-|}

	\UMLActivityControlFlow
    {init.south}{c}{--}

	\UMLActivityControlFlow
    {c}{data}{--}

  \UMLActivityControlFlow
    {c.182}{d_tr}{|-($(c)!.5!(d_tr)$)-|}

  \UMLActivityControlFlow
    {c.358}{d_meta}{|-($(c)!.5!(d_meta)$)-|}

	\UMLActivityControlFlow
    {d_tr}{tr}{--}

	\UMLActivityControlFlow
    {tr}{te}{--}

	\UMLActivityControlFlow
    {te}{d_ttsl}{--}

	\UMLActivityControlFlowWithGuard
    {d_ttsl}{saw}{--}{Not successfull}{.5}{left}

  \UMLActivityControlFlow
    {saw}{d_tr}{-|($(saw.west)-(.2,-2)$)|-}

	\UMLActivityControlFlowWithGuard
    {d_ttsl}{sdmw}{-|($(sdmw.east)+(.2,2)$)|-}
    {Successfull}{.0}{above right}

	\UMLActivityControlFlow
    {sdmw}{wptf}{--}

	\UMLActivityControlFlow
    {wptf}{kill}{--}

	\UMLActivityControlFlow
    {d_meta}{lmq}{--}

	\UMLActivityControlFlowWithGuard
    {lmq}{d_meta_one}{--}
    {Incoming Message $m$}{.5}{right}

	\UMLActivityControlFlowWithGuard
    {d_meta_one}{sew}{--}
    {$m$ contains 'env'}{.5}{right}

	\UMLActivityControlFlowWithGuard
    {d_meta_one}{mtp}{-|($(mtp.east)+(.2,2)$)|-}
    {$m$ contains 'protocol'}{.0}{above right}

	\UMLActivityControlFlow
    {mtp}{d_meta}{-|($(mtp.west)-(.2,-2)$)|-}

  \draw[blue] (sew.west) -- ($(sew.west)-(.2,0)$);

	\UMLActivityDataFlow
    {data.south}
    {data_set.north}
    {--}
    {270}
    {270}

  \UMLActivityDataFlow
    {data_set.west}
    {tr.east}
    {-|($(data_set)!.5!(tr)$)|-}
    {180}
    {180}

  % }}}
\end{tikzpicture}
\end{center}
\caption{Activity diagram of the Executor}
\end{figure}


