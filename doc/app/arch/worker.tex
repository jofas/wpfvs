\subsubsection{Worker}

The Worker $W$ is the part of the application that is
generating the data set for training the agent. The
generation of the training data set is the most expensive
task when it comes to computational effort.

While doing the generation $W$ is playing $x$ many episodes
consecutively. $x$ is statically provided by us and is
known at runtime. After finishing an episode the the score
of this episode is decisive wether the episode is good
enough for the training set, since the quality of the
training data set is very crucial for the agent to succeed.

Quality management is done statically with some constants
(like $x$).

Since generating training data is so expensive and can be
done concurrently we use many processes controlled by a
Python object called ProcessPoolExecutor from the
concurrent.futures part of Python's standard library for
this task.

\begin{figure}[H]
\begin{mdframed}[style=codebox]
\begin{lstlisting}[language=Python]
# a small example program using a ProcessPoolExecutor
from concurrent import futures

# this is executed by every process. Every process gets
# an id (i) which is used as a factor for computing a power
# sequence in the range(100*i,100*i+100)
def power_sequence(i):
  start  = 100*i
  end    = start + 100
  primes = []

  for i in range(start, end):
    primes += i**i

  return primes

# the list of powers
powers = []

# the ProcessPoolExecutor that starts 10 processes which
# means a power sequence for range 0..999 is generated
with futures.ProcessPoolExecutor(max_workers=10) as e:
  # safe every process in fs
  fs = [e.submit(power_sequence,i) for i in range(10)]

  # await the return
  for f in futures.as_completed(fs):
    powers += f.result()

print(powers)
\end{lstlisting}
\end{mdframed}
\caption{A small example program using ProcessPoolExecutor}
\end{figure}

Concurrent, multiprocessing and thread are the parts of
Python's standard library wich provide rich features for
concurrent programming.

While concurrent provides higher-level abstractions which
are more easy to use, multithreading provides the more
low-level, more powerful APIs for concurrent programming
such as a standalone Process object, Locks (mutexes),
Semaphores, Pipes, Queues or ProxyObjects
(shared variables).

We use multiprocessing for spawning standalone processes,
sharing variables and for mutexes (synchronization between
processes).


Like the Executor (cmp. \ref{s_executor}) a Worker instance
is composed of more than one process. But while an
Executor needs three, a Worker needs only two processes.

\begin{itemize}[label={}]

  \item \textbf{main process:}

        Like the Executor the Worker first initializes
        some static or shared variables and constants.
        But after the initialization part the worker has
        to get some information first before continuing
        execution.

        Since the Worker should be able to run like a
        daemon waiting for his task (provided by an
        Executor) the Worker needs the information which
        environment he should generate test data for.

        This information is provided by an Executor
        instance connected to the RabbitMQ. The Worker
        sends periodically a message to a queue the
        Executors meta process listens to. The Executor
        instance than answers with the name of the
        environment the Worker should use. The environment
        is specified by us when starting the Executor via
        its CLI.

        While it is possible to connect many Executors to
        the RabbitMQ, they all can only be started with
        the same environment since otherwise corrupt data
        will destroy any chance of success (we did not
        implement a way to distinguish between different
        environments using the same RabbitMQ).

        After having the environment the main process can
        continue.

        The main process then starts the Worker's
        generating unit called "GSL" (Generate-Send-Loop).
        This is the loop corresponding to the Executors
        "TTSL" unit.

        Followed by that the main process becomes a
        listener for a queue on the RabbitMQ. On that queue
        the main process gets the new agent provided by an
        Executor instance which is shared with the GSL
        process or a message which says that the agent
        succeeded. If that is the case the main process
        answers with a protocol to the queue the meta
        process of the Executor listens to and then kills
        the GSL process and itself.

  \item \textbf{GSL process:}

        The GSL generates and sanitizes the test data
        before sending it to the Executor. For that it uses
        Python's above mentioned ProcessPoolExecutor.

        Since the Worker is not provided with an agent yet
        it generates the actions performed in every episode
        played randomly.

        After the first batch of training data the Executor
        has processed, the Executor can send the first
        version of its agent to the Worker which can use it
        for generation afterwards.

        It should be noted that the Worker not only uses
        the agent for generating new training data but also
        spawns some processes which generate training data
        randomly so the agent does not start to make the
        same mistakes over and over again.

\end{itemize}

\begin{figure}[H]
\begin{center}
\begin{tikzpicture}

  \UMLActivitySwimlane{GSL}{\linewidth/2}{\textheight-5cm}
    {}
  \UMLActivitySwimlane{main}{\linewidth/2}{\textheight-5cm}
    {right=0 of TTSL}


  \UMLActivityInitialNodeRelativeTo{above=.5 of main}


  \UMLActivityConcurrentNodeHRelativeTo
    {above=4 of main.west}
    {c}
    {12cm}


  % main {{{
	\UMLActivityStateRelativeToAlterName
    {above=-3 of main}
    {Initialize global shared variables and constants}
    {init}
    {text width=4cm}

	\UMLActivityStateRelativeToAlterName
    {below=1 of init}
    {wait for environment}
    {woe}
    {text width=4cm}

  \UMLActivityStateRelativeToAlterName
    {below=3 of woe}
    {Listening for new agent}
    {lna}
    {text width=4cm}

  \UMLActivityCentralBufferRelativeToAlterName
    {below=2 of lna}
    {Agent}
    {a}
    {text width=4cm}

  \UMLActivityStateRelativeToAlterName
    {below=2 of a}
    {Send protocol}
    {sp}
    {text width=4cm}

  \UMLActivityExitNodeRelativeTo
    {below=2 of sp}
    {kill}

  % }}}

  % gsl {{{
  \UMLActivityDescisionNodeRelativeTo
    {above=-9 of GSL}
    {d_g}

  \UMLActivityStateRelativeToAlterName
    {below=2 of d_g}
    {Generating}
    {g}
    {text width=4cm}

  \UMLActivityStateRelativeToAlterName
    {below=2 of g}
    {Sending}
    {s}
    {text width=4cm}
  % }}}

  % conns {{{
  \UMLActivityControlFlow
    {initial.east}{init}
    {-|($(init.north)+(3.5,0.3)$)|-}

	\UMLActivityControlFlow
    {init.south}{woe}{--}

	\UMLActivityControlFlow
    {woe}{c.2}{|-($(woe)!.5!(c)$)-|}

	\UMLActivityControlFlow
    {c.358}{lna.north}{|-($(c)!.5!(lna)$)-|}

	\UMLActivityControlFlowWithGuard
    {lna.east}{sp.east}{-|($(lna.east)+(.5,-2)$)|-}
    {Executioner sends done message}{.0}
    {above right, text width=2cm}

	\UMLActivityControlFlow
    {sp}{kill}{--}

  \UMLActivityControlFlow
    {c.182}{d_g}{|-($(c)!.5!(d_g)$)-|}

	\UMLActivityControlFlow
    {d_g}{g}{--}

	\UMLActivityControlFlow
    {g}{s}{--}

  \UMLActivityControlFlow
    {s}{d_g}{-|($(s.west)-(.5,-2)$)|-}

	\UMLActivityDataFlow
    {lna.south}
    {a.north}
    {--}
    {270}
    {270}
  \node[left] at ($(lna)!.5!(a)$)
    {\tiny{[Executioner sends agent]}};

  \UMLActivityDataFlow
    {a.west}
    {g.east}
    {-|($(a)!.5!(g)$)|-}
    {180}
    {180}
  % }}}
\end{tikzpicture}
\end{center}
\caption{Activity diagram of the Worker}
\end{figure}


