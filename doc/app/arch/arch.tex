\subsection{Architecture}

This chapter will describe in detail how our application
is build, how it works and what we use to achieve
concurrency.

On a higher level our application is devided into two
distinct parts, the Executor ($E$) and the Worker ($W$).

$E$ is the part of our application that is doing the
training and testing of our agent, while  $W$ generates
training data which is sent via the RabbitMQ to $E$ so $E$
can train the agent with this data. After training $E$
sends the agent to $W$ so $W$ can generate new training
data with the updated agent.

This iteration is continued until the agent is able to
solve the Gym environment (cmp. \ref{s_openai_gym}).

% Network {{{
\subsubsection{Network}

Instances of both, $E_i$ and $W_j$ communicate over a
message broker, in this case RabbitMQ (cmp.
\ref{s_message_broker}, \ref{s_rabbitmq}).

\begin{figure}[H]
\begin{center}
\begin{tikzpicture}

  \node[inner sep=15pt]
    at (0,0) (rmq) {};

  \node [label=$W_2$, inner ysep=50pt, above=4cm of rmq]
    (w_two) {};
  \draw[-{Stealth[color=white,length=10mm]}]
    (w_two.center) -- (rmq);
  \PCIcon{w_two}{scale=.25}

  \node [label=$W_1$, inner ysep=50pt, left =2cm of w_two]
    (w_one)   {};
  \draw[-{Stealth[color=white,length=10mm]}]
    (w_one.center) -- (rmq);
  \PCIcon{w_one}{scale=.25}

  \node [label=$W_0$, inner ysep=50pt, left =2cm of w_one]
    (w_zero)   {};
  \draw[-{Stealth[color=white,length=10mm]}]
    (w_zero.center) -- (rmq);
  \PCIcon{w_zero}{scale=.25}

  \node [right=2cm of w_two] (w_dots) {\dots};

  \node [label=$W_n$, inner ysep=50pt, right=2cm of w_dots]
    (w_n) {};
  \draw[-{Stealth[color=white,length=10mm]}]
    (w_n.center) -- (rmq);
  \PCIcon{w_n}{scale=.25}


  \node [label={below:$E_2$}, inner ysep=40pt,
    below=4cm of rmq]
    (e_two) {}
    edge[-{Stealth[color=white,length=10mm]}] (rmq);
  \PCIcon{e_two}{scale=.25}

  \node [label={below:$E_1$}, inner ysep=40pt,
    left =2cm of e_two]
    (e_one)   {}
    edge[-{Stealth[color=white,length=10mm]}] (rmq);
  \PCIcon{e_one}{scale=.25}

  \node [label={below:$E_0$}, inner ysep=40pt,
    left =2cm of e_one]
    (e_zero)   {}
    edge[-{Stealth[color=white,length=10mm]}] (rmq);
  \PCIcon{e_zero}{scale=.25}

  \node [right=2cm of e_two] (e_dots) {\dots};

  \node [label={below:$E_m$}, inner ysep=40pt,
    right=2cm of e_dots]
    (e_n) {}
    edge[-{Stealth[color=white,length=12mm]}] (rmq);
  \PCIcon{e_n}{scale=.25}


  \node[label={below:RabbitMQ},inner ysep=15pt]
    at (0,0) {};
  \RabbitMQ{rmq}{scale=.25}

\end{tikzpicture}
\end{center}
\caption{Architecture of our Application on the network
  level}
\end{figure}

% }}}


\subsubsection{Executor}
\label{s_executor}
\index{Application!Executor}

The Executor $E$ is the part of the application that is
doing the training and testing of our agent. Training and
testing is done incrementally in a loop, the "TTSL"
(Training-Testing-Sending-Loop). An Executor instance
$E_i$ gets provided with its training data from the Worker
instances $W_j$ it is connected to (via the RabbitMQ).

If the testing part of the "TTSL" failed $E_i$ sends the
newly trained agent to every $W_j$ (the sending part of the
"TTSL"). Else if the testing was successful and the agent
mastered the environment (cmp \ref{s_openai_gym}) it sends
a message that testing was successful which kills every
$W_j$.

The Executor is composed of three processes, the main
process, the meta process and the TTSL process.

\begin{itemize}[label={}]

  \item \textbf{main process:}

        The main process first initializes global shared
        variables and constants it shares with the other
        two processes. Then it starts the meta process and
        the TTSL process.

        After that the main process becomes a listener
        which listens for incoming data from the Worker
        instances connected to the RabbitMQ. If new data
        comes in it is saved in a shared variable so the
        TTSL process can access it.
        \index{Application!Executor!main process}

  \item \textbf{TTSL process:}

        The process executing the TTSL.
        \index{Application!Executor!TTSL process}

  \item \textbf{meta process:}

        This process is a listener (like main becomes after
        initializing the shared memory and starting this
        and the TTSL process) which listens to a queue
        of the RabbitMQ so this Executor can communicate
        with the Workers it is connected to.
        \index{Application!Executor!meta process}

\end{itemize}

\begin{figure}[H]
\begin{center}
\begin{tikzpicture}

  \UMLActivitySwimlane{TTSL}{\linewidth/3}{\textheight-5cm}
    {}
  \UMLActivitySwimlane{main}{\linewidth/3}{\textheight-5cm}
    {right=0 of TTSL}
  \UMLActivitySwimlane{meta}{\linewidth/3}{\textheight-5cm}
    {right=0 of main}



  \UMLActivityInitialNodeRelativeTo{above=.5 of main}


  % main {{{
	\UMLActivityStateRelativeToAlterName
    {above=-3 of main}
    {Initialize global shared variables and constants}
    {init}
    {text width=4cm}

  \UMLActivityConcurrentNodeHRelativeTo
    {below=1 of init}
    {c}
    {16cm}

	\UMLActivityStateRelativeToAlterName
    {below=1 of c}
    {Listening for new training data}
    {data}
    {text width=4cm}

  \UMLActivityCentralBufferRelativeToAlterName
    {below=1 of data}
    {Training data set}
    {data_set}
    {text width=4cm}
  % }}}

  % ttsl{{{
  \UMLActivityDescisionNodeRelativeTo
    {above=-6 of TTSL}
    {d_tr}

  \UMLActivityStateRelativeToAlterName
    {below=1 of d_tr}
    {Training}
    {tr}
    {text width=4cm}

  \UMLActivityStateRelativeToAlterName
    {below=1 of tr}
    {Testing}
    {te}
    {text width=4cm}

  \UMLActivityDescisionNodeRelativeTo
    {below=1 of te}
    {d_ttsl}

  \UMLActivityStateRelativeToAlterName
    {below=1 of d_ttsl}
    {Send agent to Worker}
    {saw}
    {text width=4cm}

  \UMLActivityStateRelativeToAlterName
    {below=1 of saw}
    {Send done message to Worker}
    {sdmw}
    {text width=4cm}

  \UMLActivityStateRelativeToAlterName
    {below=1 of sdmw}
    {Write protocol to file}
    {wptf}
    {text width=4cm}

  \UMLActivityExitNodeRelativeTo
    {below=1 of wptf}
    {kill}
  % }}}

  % meta {{{
  \UMLActivityDescisionNodeRelativeTo
    {above=-6 of meta}
    {d_meta}

  \UMLActivityStateRelativeToAlterName
    {below=1 of d_meta}
    {Listening to META\-QUEUE}
    {lmq}
    {text width=4cm}

  \UMLActivityDescisionNodeRelativeTo
    {below=1 of lmq}
    {d_meta_one}

  \UMLActivityStateRelativeToAlterName
    {below=1 of d_meta_one}
    {Send environment to Worker}
    {sew}
    {text width=4cm}

  \UMLActivityStateRelativeToAlterName
    {below=1 of sew}
    {Add message to the protocol}
    {mtp}
    {text width=4cm}
  % }}}

  % conns {{{
	\UMLActivityControlFlow
    {initial.east}{init.north}
    {-|(8,10)|-($(init.north)+(0,0.3)$)-|}

	\UMLActivityControlFlow
    {init.south}{c}{--}

	\UMLActivityControlFlow
    {c}{data}{--}

  \UMLActivityControlFlow
    {c.182}{d_tr}{|-($(c)!.5!(d_tr)$)-|}

  \UMLActivityControlFlow
    {c.358}{d_meta}{|-($(c)!.5!(d_meta)$)-|}

	\UMLActivityControlFlow
    {d_tr}{tr}{--}

	\UMLActivityControlFlow
    {tr}{te}{--}

	\UMLActivityControlFlow
    {te}{d_ttsl}{--}

	\UMLActivityControlFlowWithGuard
    {d_ttsl}{saw}{--}{Not successfull}{.5}{left}

  \UMLActivityControlFlow
    {saw}{d_tr}{-|($(saw.west)-(.2,-2)$)|-}

	\UMLActivityControlFlowWithGuard
    {d_ttsl}{sdmw}{-|($(sdmw.east)+(.2,2)$)|-}
    {Successfull}{.0}{above right}

	\UMLActivityControlFlow
    {sdmw}{wptf}{--}

	\UMLActivityControlFlow
    {wptf}{kill}{--}

	\UMLActivityControlFlow
    {d_meta}{lmq}{--}

	\UMLActivityControlFlowWithGuard
    {lmq}{d_meta_one}{--}
    {Incoming Message $m$}{.5}{right}

	\UMLActivityControlFlowWithGuard
    {d_meta_one}{sew}{--}
    {$m$ contains 'env'}{.5}{right}

	\UMLActivityControlFlowWithGuard
    {d_meta_one}{mtp}{-|($(mtp.east)+(.2,2)$)|-}
    {$m$ contains 'protocol'}{.0}{above right}

	\UMLActivityControlFlow
    {mtp}{d_meta}{-|($(mtp.west)-(.2,-2)$)|-}

  \draw[blue] (sew.west) -- ($(sew.west)-(.2,0)$);

	\UMLActivityDataFlow
    {data.south}
    {data_set.north}
    {--}
    {270}
    {270}

  \UMLActivityDataFlow
    {data_set.west}
    {tr.east}
    {-|($(data_set)!.5!(tr)$)|-}
    {180}
    {180}

  % }}}
\end{tikzpicture}
\end{center}
\caption{Activity diagram of the Executor}
\end{figure}




\subsubsection{Worker}
\label{s_worker}
\index{Application!Worker}

The Worker $W$ is the part of the application that is
generating the data set for training the agent. The
generation of the training data set is the most expensive
task when it comes to computational effort.

While doing the generation $W$ is playing $x$ many episodes
consecutively. $x$ is statically provided by us and is
known at runtime. After finishing an episode the the score
of this episode is decisive wether the episode is good
enough for the training set, since the quality of the
training data set is very crucial for the agent to succeed.

Quality management is done statically with some constants
(like $x$).

Since generating training data is so expensive and can be
done concurrently we use many processes controlled by a
Python object called ProcessPoolExecutor from the
concurrent.futures part of Python's standard library for
this task.
\index{ProcessPoolExecutor}

\begin{figure}[H]
\begin{mdframed}[style=codebox]
\begin{lstlisting}[language=Python]
# a small example program using a ProcessPoolExecutor
from concurrent import futures

# this is executed by every process. Every process gets
# an id (i) which is used as a factor for computing a power
# sequence in the range(100*i,100*i+100)
def power_sequence(i):
  start  = 100*i
  end    = start + 100
  _powers = []

  for i in range(start, end):
    _powers += i**i

  return _powers

# the list of powers
powers = []

# the ProcessPoolExecutor that starts 10 processes which
# means a power sequence for range 0..999 is generated
with futures.ProcessPoolExecutor(max_workers=10) as e:
  # safe every process in fs
  fs = [e.submit(power_sequence,i) for i in range(10)]

  # await the return
  for f in futures.as_completed(fs):
    powers += f.result()

print(powers)
\end{lstlisting}
\end{mdframed}
\caption{A small example program using ProcessPoolExecutor}
\end{figure}

Concurrent, multiprocessing and thread are the parts of
Python's standard library wich provide rich features for
concurrent programming.

While concurrent provides higher-level abstractions which
are more easy to use, multithreading provides the more
low-level, more powerful APIs for concurrent programming
such as a standalone Process object, Locks (mutexes),
Semaphores, Pipes, Queues or ProxyObjects
(shared variables).

We use multiprocessing for spawning standalone processes,
sharing variables and for mutexes (synchronization between
processes).

Like the Executor (cmp. \ref{s_executor}) a Worker instance
is composed of more than one process. But while an
Executor needs three, a Worker needs only two processes.

\begin{itemize}[label={}]

  \item \textbf{main process:}

        Like the Executor the Worker first initializes
        some static or shared variables and constants.
        But after the initialization part the worker has
        to get some information first before continuing
        execution.

        Since the Worker should be able to run like a
        daemon waiting for his task (provided by an
        Executor) the Worker needs the information which
        environment he should generate test data for.

        This information is provided by an Executor
        instance connected to the RabbitMQ. The Worker
        sends periodically a message to a queue the
        Executors meta process listens to. The Executor
        instance than answers with the name of the
        environment the Worker should use. The environment
        is specified by us when starting the Executor via
        its CLI.

        While it is possible to connect many Executors to
        the RabbitMQ, they all can only be started with
        the same environment since otherwise corrupt data
        will destroy any chance of success (we did not
        implement a way to distinguish between different
        environments using the same RabbitMQ).

        After having the environment the main process can
        continue.

        The main process then starts the Worker's
        generating unit called "GSL" (Generate-Send-Loop).
        This is the loop corresponding to the Executors
        "TTSL" unit.

        Followed by that the main process becomes a
        listener for a queue on the RabbitMQ. On that queue
        the main process gets the new agent provided by an
        Executor instance which is shared with the GSL
        process or a message which says that the agent
        succeeded. If that is the case the main process
        answers with a protocol to the queue the meta
        process of the Executor listens to and then kills
        the GSL process and itself.
        \index{Application!Worker!main process}

  \item \textbf{GSL process:}

        The GSL generates and sanitizes the test data
        before sending it to the Executor. For that it uses
        Python's above mentioned ProcessPoolExecutor.

        Since the Worker is not provided with an agent yet
        it generates the actions performed in every episode
        played randomly.

        After the first batch of training data the Executor
        has processed, the Executor can send the first
        version of its agent to the Worker which can use it
        for generation afterwards.

        It should be noted that the Worker not only uses
        the agent for generating new training data but also
        spawns some processes which generate training data
        randomly so the agent does not start to make the
        same mistakes over and over again.
        \index{Application!Worker!GSL process}

\end{itemize}

\begin{figure}[H]
\begin{center}
\begin{tikzpicture}

  \UMLActivitySwimlane{GSL}{\linewidth/2}{\textheight-5cm}
    {}
  \UMLActivitySwimlane{main}{\linewidth/2}{\textheight-5cm}
    {right=0 of TTSL}


  \UMLActivityInitialNodeRelativeTo{above=.5 of main}


  \UMLActivityConcurrentNodeHRelativeTo
    {above=4 of main.west}
    {c}
    {12cm}


  % main {{{
	\UMLActivityStateRelativeToAlterName
    {above=-3 of main}
    {Initialize global shared variables and constants}
    {init}
    {text width=4cm}

	\UMLActivityStateRelativeToAlterName
    {below=1 of init}
    {wait for environment}
    {woe}
    {text width=4cm}

  \UMLActivityStateRelativeToAlterName
    {below=3 of woe}
    {Listening for new agent}
    {lna}
    {text width=4cm}

  \UMLActivityCentralBufferRelativeToAlterName
    {below=2 of lna}
    {Agent}
    {a}
    {text width=4cm}

  \UMLActivityStateRelativeToAlterName
    {below=2 of a}
    {Send protocol}
    {sp}
    {text width=4cm}

  \UMLActivityExitNodeRelativeTo
    {below=2 of sp}
    {kill}

  % }}}

  % gsl {{{
  \UMLActivityDescisionNodeRelativeTo
    {above=-9 of GSL}
    {d_g}

  \UMLActivityStateRelativeToAlterName
    {below=2 of d_g}
    {Generating}
    {g}
    {text width=4cm}

  \UMLActivityStateRelativeToAlterName
    {below=2 of g}
    {Sending}
    {s}
    {text width=4cm}
  % }}}

  % conns {{{
  \UMLActivityControlFlow
    {initial.east}{init}
    {-|($(init.north)+(3.5,0.3)$)|-}

	\UMLActivityControlFlow
    {init.south}{woe}{--}

	\UMLActivityControlFlow
    {woe}{c.2}{|-($(woe)!.5!(c)$)-|}

	\UMLActivityControlFlow
    {c.358}{lna.north}{|-($(c)!.5!(lna)$)-|}

	\UMLActivityControlFlowWithGuard
    {lna.east}{sp.east}{-|($(lna.east)+(.5,-2)$)|-}
    {Executioner sends done message}{.0}
    {above right, text width=2cm}

	\UMLActivityControlFlow
    {sp}{kill}{--}

  \UMLActivityControlFlow
    {c.182}{d_g}{|-($(c)!.5!(d_g)$)-|}

	\UMLActivityControlFlow
    {d_g}{g}{--}

	\UMLActivityControlFlow
    {g}{s}{--}

  \UMLActivityControlFlow
    {s}{d_g}{-|($(s.west)-(.5,-2)$)|-}

	\UMLActivityDataFlow
    {lna.south}
    {a.north}
    {--}
    {270}
    {270}
  \node[left] at ($(lna)!.5!(a)$)
    {\tiny{[Executioner sends agent]}};

  \UMLActivityDataFlow
    {a.west}
    {g.east}
    {-|($(a)!.5!(g)$)|-}
    {180}
    {180}
  % }}}
\end{tikzpicture}
\end{center}
\caption{Activity diagram of the Worker}
\end{figure}




\subsubsection{Queues and Exchanges}

Now, this chapter will go more into detail on how we
utilize the RabbitMQ.

First, there are two concepts we used for communication
with the RabbitMQ called queues and exchanges.

\begin{itemize}[label={}]

  \item \textbf{queue:}

        This concept we already discussed in chapter
        \ref{ss_mb_faa}. Now, for our project one property
        of a queue is interesting, which is "first come,
        first serve" or "FIFO" (First In, First Out), which
        means once a message is received by a listener
        (also called consumer), other listeners (consumers)
        that subscribed to this particular queue will never
        receive this message and all subscribed listeners
        are in a race condition for the next message.

        Because of this we need another concept for
        distributing certain messages (e.g. sending our
        agent to each Worker, since every Worker should use
        a current version of the agent wich would not be
        possible if a Executor would send the agent to a
        queue, because then only one would receive the new
        agent instead of all Workers).

  \item \textbf{exchange:}

        For the above mentioned szenario we need a
        different message distribution called
        publish/subscribe.

        Publish/subscribe message distribution can be
        achieved with an exchange provided by the RabbitMQ.

        In this szenario the Executor would by the
        publisher while the Workers would be the
        subscribers. For every listener (subscriber)
        RabbitMQ generates a new queue and redistributes
        every message published to the exchange to the
        queues.

        For this redistribution or routing are some methods
        available. We only used the method called fanout,
        which generates a copy of every published message
        for each listener (consumer) receiving on an
        exchange.

\end{itemize}

\begin{figure}[H]
\begin{center}
\begin{tikzpicture}

\node[rectangle split, rectangle split parts = 5,
      rectangle split ignore empty parts=false,
      rectangle split horizontal,
      inner sep=11pt, draw, label=queue]
  at (0,0) (queue) {};

\node[circle,fill=red!25,left=9.5 of queue,label=publisher]
  {P}
  edge[->] (queue);

\node[circle,fill=green!25,right=2 of queue,label=listener]
  {L}
  edge[<-] (queue);



\node[rectangle split, rectangle split parts = 5,
      rectangle split ignore empty parts=false,
      rectangle split horizontal,
      inner sep=11pt, draw]
  at (0,-8) (queue_two) {};

\node[rectangle split, rectangle split parts = 5,
      rectangle split ignore empty parts=false,
      rectangle split horizontal,
      inner sep=11pt, draw, above=1.5 of queue_two]
  (queue_one) {};

\node[rectangle split, rectangle split parts = 5,
      rectangle split ignore empty parts=false,
      rectangle split horizontal,
      inner sep=11pt, draw, above=1.5 of queue_one]
  (queue_zero) {};

\node[below=1 of queue_two] (queue_dots) {\vdots};

\node[rectangle split, rectangle split parts = 5,
      rectangle split ignore empty parts=false,
      rectangle split horizontal,
      inner sep=11pt, draw, below=1.5 of queue_dots]
  (queue_n) {};


\node[circle,fill=blue!25,left=6.5 of queue_two,
      label=exchange]
  (exchange) {E}
  edge[->] (queue_two.west)
  edge[->] (queue_one.west)
  edge[->] (queue_zero.west)
  edge[->] (queue_n.west);


\node[circle,fill=red!25,left=2 of exchange,
      label=publisher]
  {P}
  edge[->] (exchange);


\node[circle,fill=green!25,right=2 of queue_two,
      label=listener]
  (l_two) {L$_2$}
  edge[<-] (queue_two.east);

\node[circle,fill=green!25,label=listener,
      above=1.5 of l_two]
  (l_one) {L$_1$}
  edge[<-] (queue_one.east);

\node[circle,fill=green!25,right=2 of queue,label=listener,
      above=1.5 of l_one]
  {L$_0$}
  edge[<-] (queue_zero.east);

\node[below=1 of l_two] (l_dots) {\vdots};

\node[circle,fill=green!25,right=2 of queue,
      label={below:listener},
      below=1.5 of l_dots]
  {L$_n$}
  edge[<-] (queue_n.east);

\end{tikzpicture}
\end{center}
\caption{Scheme of a queue and an exchange}
\end{figure}


The queues and exchanges we used in our project:

\begin{itemize}[label={}]

  \item \textbf{meta\_queue:}

        Queue the meta process of the Executor(s) is/are
        listening to. This queue is used by Workers to
        ask for the environment and for sending the
        protocol once the agent succeeded.

  \item \textbf{meta\_exchange:}

        Corresponds to meta\_queue. The Executor(s) is/are
        answering with the environment on this exchange.

  \item \textbf{data\_queue:}

        Queue used by the Workers to send their generated
        training data to the Executor(s).

  \item \textbf{model\_exchange:}

        Exchange the Executor(s) use(s) for sending the new
        agent to each Worker or, if the agent succeeded,
        for sending a message telling each Worker to send
        their protocols and shut down afterwards.

\end{itemize}

