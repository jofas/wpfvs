% The agent {{{
\subsection{The agent}
\label{s_agent}

Before going into detail on the application's architecture,
here is a brief summary on what our agent actually is.

We programmed our agent as a neural network with the Keras
library, which has an API for high level, high abstraction
neural networks.

Keras uses Tensorflow as its backend for computations and
basically only provides a nicer abstraction of Tensorflow.

% example {{{
\begin{mdframed}[style=codebox]
\begin{lstlisting}[language=Python]
# a small example program using Keras
#
# API documentation at: https://keras.io/

# Sequential is the keras object representing a neural net-
# work
from keras.models import Sequential
# a neural net is comprised of layers connected with each
# other. The Dense object represents a layer
from keras.layers import Dense

# the neural net. This specific neural net has four layers
# (the input (size of 12), two hidden (both 64 artificial
# neurons) and the output layer (size of 4)). The input
# layer does not have to be specified since the input_dim
# parameter of the first layer automatically generates the
# input layer.
model = Sequential([
  Dense(64, activation='relu', input_dim=12 ),
  Dense(64, activation='relu'               ),
  Dense(4,  activation='softmax'            ),
])

# define how the model should learn and some other meta
# information for the training process (learning algorithm,
# optimizer, etc)
model.compile(
        optimizer = 'adam',
        loss      = 'categorical_crossentropy',
        metrics   = ['accuracy']
)

# generate a dummy data set with corresponding labels with
# 1000 entries for training
import numpy as np

data   = np.random.random((1000,12))
# generating the labels (either a 0 or a 1)
labels = np.random.randint(2, size=(1000, 1))

# training the model, iterating 10 times
model.train(data, labels, epochs=10)

# using the neural net to predict the label of a random
# data point
test   = np.random.random((1,12))

model.predict(test)
\end{lstlisting}
\end{mdframed}
\begin{figure}[H]
\caption{A small example program using Keras}
\end{figure}
% }}}

Our agent has the observation provided by the Gym
environment (cmp. \ref{s_openai_gym}) as its input
parameters and as output parameters the actions (the size
of the output layer equals the action space $a$
($|[0..a[| = a$) of the Gym environment.

% }}}
